The aim of our project is to create and maintain corpora for machine translation without the need of hiring expensive translators while using the power of crowd instead. We achieve this with online interface which allows addition of new phrases to corpus their translation as well as evaluation of the best translation. Aside from creating the corpus application offers language practice and access to foreign content to users.

Our Proposal
In general there are three steps in work cycle of such application. Addition of new phrases their translation and finally their evaluation. Our solution to first step is to use social network Twitter. Application periodically scans the network for hashtag #tctrq and adds content of such tweets to database. Each tweet is required to contain another hashtag with the target language of translation. Twitter uses it's own system to determine language of each tweet and our application uses this information. 
Second step the translation takes step immediately thereafter. Each of registered translators capable of translation between source and target language is notified via e-mail and submits translation as a reply e-mail. These replies are periodically collected and added to database. At this point machine translation from Moses is added to compete with human translators. 
Third step of work cycle the evaluation is the only step that requires user to come to our website and use simple interface to vote between two translations. Elo rating system is used to evaluate the results. Voting is of course blinded. After gaining high enough score the translation is posted back to twitter as a response to requester and thus completing the cycle. 

Our solution
The application is developed with the CakePHP framework. For all e-mail communication we use associated Gmail account through IMAP protocol. The Twitter integration is done with Simple PHP Wrapper for Twitter through REST API.

Moses - http://www.statmt.org/moses/
Twitter - http://twitter.com
CakePHP - http://cakephp.org/
Simple PHP Wrapper for Twitter API v1.1 calls - http://github.com/J7mbo/twitter-api-php
